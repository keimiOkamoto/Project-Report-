\documentclass[a4paper, 11pt]{article}

%\usepackage[parfill]{parskip}
\usepackage{ragged2e}
\usepackage{graphicx}
\graphicspath{ {images/} }
\usepackage[T1]{fontenc}

\newlength{\drop}
\usepackage{epigraph}
\usepackage{dirtytalk}
\usepackage{wrapfig}
\usepackage{quoting}

%\usepackage[pdftex,active,tightpage]{preview} 
%\setlength\PreviewBorder{2mm} 
\usepackage{gantt}
\renewcommand{\epigraphflush}{center}
\renewcommand{\epigraphwidth}{1\textwidth}

%........................
% Header/Footers
%........................
\usepackage{lastpage}
\usepackage{fancyhdr}

%%%%%%%START%%%%%%%
\begin{document}
  \begin{titlepage}
	\thispagestyle{empty}
    \drop=0.1\textheight
    \centering
    \vspace*{\baselineskip}
    \rule{\textwidth}{1.6pt}\vspace*{-\baselineskip}\vspace*{2pt}
    \rule{\textwidth}{0.4pt}\\[\baselineskip]
    {\Large{MSc Computer Science\\[0.3\baselineskip] }} 	
    {\huge{Project Report\\[0.3\baselineskip] }}
	
    \rule{\textwidth}{0.4pt}\vspace*{-\baselineskip}\vspace{3.2pt}
    \rule{\textwidth}{1.6pt}
    \\[\baselineskip]
    \scshape
    {\Large Ubiquitous Consumer Inventory Management System For Waste Prevention\\}
%    Location, date from--to\par
    \vspace*{2\baselineskip}
    %Edited by \\[\baselineskip]
    {\normalsize\emph{Supervisor: }{\large Professor George Roussos\par}}
    {\normalsize\emph{Author: }{\large Keimi Okamoto\par}}
    
    {\itshape 2015}
    \vfill
    {\large BIRKBECK UNIVERSITY OF LONDON\par}
{\footnotesize DEPARTMENT OF COMPUTER SCIENCE \& INFORMATION SYSTEMS}\par
  \end{titlepage}
  
%\maketitle

%........................
% Contents
%.......................
\pagenumbering{roman}
\tableofcontents
\clearpage

\clearpage



%........................
% Introduction
%........................

\pagestyle{fancy}
\renewcommand{\sectionmark}[1]{\markboth{#1}{}}
\fancyhf{} % sets both header and footer to nothing
\renewcommand{\headrulewidth}{0pt}
\setlength{\footskip}{80pt}
\lhead{}
\chead{}
\rhead{Project Report}
\pagenumbering{arabic}
\lfoot{Section \thesection}
\cfoot{\fancyplain{}{\leftmark }} 
\rfoot{\thepage\ of \pageref{LastPage}}

\setcounter{page}{1}
\section{Introduction}

\subsection{Abstract}
What the project is about in regards to the proposal. The problem i am trying to solve. Over production food.
Tackeling the core, If the consumers stop over purchasing the supermarkets will stop over producing. 


\subsection{Structure of the report}
Present the reader with an overview of each chapter.

\subsection{Development Methodology}
UML and agile. And the reasons why this is a productive methodology.
Version control for documentation and history. 
\clearpage

%..................................................
% Background
%..................................................
\section{Background \& Analysis}
This section presents the problem and the intent for the project supported by the analysis of currently available processes. 

\subsection{Motivation}
The intention of the system is to aid the reduction in food waste created by households. Below highlight the motivation for the project and the cause of waste. 

\subsubsection{Environment}
Globally food waste has exceeded to two billion tonnes annually. Spoiling food emits methane, a particularly harmful greenhouse gas contributing to the warming of the earth's surface and threatening the echo-system. Due to human activity methane gas being released into the atmosphere has increased two and a half folds since the industrial revolution. Further more, resources used during the rearing and plantation process for the produce, as well as the use of packaging and transportation all contribute to the detrimental effect on our environment. Local councils are continuously pressured to fuel more funds into waste management recourses, landfill sites are overflowing and space to accommodate the waste is fast diminishing.

In the United Kingdom alone consumers and households are responsible for at least seven million tonnes of waste, whilst in comparison the retailers contribute a mere two hundred and twenty five thousand tonnes. Although the reported figures implicate the household as the largest culprit, retailers are in fact the catalyst force behind the mass generation of waste. Frequent buy-one-get-one-free and multi-buy discounts hosted by retailers contribute largely to the problem. Competition for market share is fierce between the retailers and goods must always be available at a lower price than that of their neighbours. Farmers and other producers alike are pressured to over produce to accommodate for a sudden rise in demand or to simply cover the risk of a potentially bad harvest. But when the sales forecast is not met or the harvest is overly fruitful, supermarkets will routinely deduct the price of food nearing expiration as a method of damage control for their investment. 

Consumers are enticed by the attractive offer of a free item after purchasing two, and encouraged to over purchase food that they do not need and will most likely not consume before the use-by-date. Thus, waste that was originally created by the retailers is pushed down the chain, residing with the consumers. This aggressive sales strategy not only causes monetary waste to the individual but also engineers the blind participation in driving up demands in goods, in turn encouraging the overproduction of food. 


\subsubsection{Public Health}
In addition to the environmental damage, there are growing concerns over the public health risks the bargains offers bring. Over purchasing of food can encourage excessive consumption. Discounted foods usually come with a shorter use-by-date, meaning over a shorter span of time an individual must consume more than necessary in order for their investment to be justified. This side effect is detrimental to the health and well being of the public as the national statistics report for the United Kingdom has unveiled. Obesity rates in male adults have increased 13.2\% and 7.4\% for female adults since 1993. These figures are rising every year and the World Health Organisation (WHO) has predicted that by 2030 74\% of male adults and 64\% of female adults will be obese if precautionary measures are not implemented. 

\subsubsection{Aiding the modern lifestyle}
The final problem is simply human errors. We purchase produce with the good intention of consuming them. But as the demands of our fast paced modern life-style take priority, we often forget all about the existence of what we stocked and inadvertently let our stock expire until it is no longer fit for consumption. Expiry dates for different produces vary, and keeping track of all dates is a near impossible task by relying on human memory alone. A popular study carried out in by George Armitage Miller, a prominent figure in the field of cognitive psychology discovered that the number of objects an average human can hold in working memory is seven, give or take two. With this limited capacity is no surprise that once the fridge door is closed, it is inevitable that some of the produce is destined for the waste bin.

Many homes are made up of multiple inhabitants. Errors such as double purchasing due to lack of communication is a common occurrence. If two occupiers notice an item is low on stock at different times they may both set out to replace the item, resulting in duplicate items being purchased and increasing the risk of waste occurring. 


\vspace{\baselineskip}
\vspace{\baselineskip}
\subsection{Current Waste Reduction Methods}
Governments and organisations have implemented various measures to tackle the rising figures in waste. Below describe the techniques used and the strengths and weaknesses that each possess. 

\subsubsection{Manual Labour} 
Such efforts include manual labour by distributing flyers aimed to inform and educate individuals if the implications of waste. Engaging with communities by setting up stalls and public demonstrations to raise awareness has been a favourable method for generating interest. This form of engagement is inspirational and informative but an expensive operation; sustaining the interest of the community is tough and effects are short lived.

\subsubsection{Anaerobic Digestion} 
Waste management organisations such as Biffa and the retail giant Sainsbury's have collaborated to recycle food waste using anaerobic digestion. By utilising the gases produced from the waste they are able to generate power to sustain the running of retail stores. Anaerobic digestion creates a circular process where the waste produced by the retailers is pumped back into the production line to fuel the very instrument creating the problem. This method has been questioned as to whether it is a true solution as the resources used throughout the process also leave a carbon footprint.

\subsubsection{Smart Fridge} Smart fridges were introduced in the early 2000's as a home inventory management system. Designed to be integrated with our every day lives and to monitor the inhabitants investment. The user inputs the product on either an embedded screen or mobile device. The fridge was designed to keep track and inform the owner if items are running low on stock and support the automatic replenishment of goods. The primary hindrance to sales of this product is largely attributed to cost and lack of infrastructure. 

\subsubsection{Nano Technology} 
Most recently, Nano technology has been used to monitor the stages of decomposition in foods. A small gel like cube emits a colour corresponding to a particular stage of decomposition and visually representing the shelf life of the product. Nano technology is able to remove the need for printed sell-by-dates but it still requires the inhabitants to actively open the fridge door, view, process and memorise the colours of the tags for various products. 

\subsubsection{Mobile Application} 
Mobile phone applications have been created by charity-funded organisations such as Love Food Hate Waste. This application informs the user of the recommended portion to discourage over purchasing. Other features include recipe recommendations and a shopping list creator and inventory manager. 
Apps such Love Food Hate Waste and other product login applications all suffer from the same bottleneck, that act of having to manually punch in the product details. This arduous process dissuades many, and the application is soon discarded. Some applications have implemented barcode scanning using the phones camera but if the barcode is damaged in the slightest it is rendered unreadable. 

\subsubsection{Radio Frequency Identification} 
Radio frequency identification (RFID) has been utilised in the food supply chain for many years for inventory tracking. Recently Dutch researchers have developed sensor enabled tag, the tag monitors atmospherical changes in the environment the produce is exposed to during transit. By analysing the data collected with the Pasteur sensor tag an estimate shelf life for the produce is generated thus reducing the likelihood of waste occurring. Currently this technology is only available for large-scale shipments and used only to protect the transportation between producers to retailers. 

\subsection{Proposed Approach to the Problem}
In attempts to reduce the generation of food waste, in the past the British government has intervened and officials pressured retailers to abolish the multi-buy offers. But the suggestion was met with reluctance. Instead a compromise was reached and the revised promotions allowed consumers a wider variety of products to choose from. Although figures declined periodically it was not sustainable. It is evident from this that persuading retailers to prioritise the reduction of waste over the potential increase in revenue is an up hill struggle. Without constant monitoring, retailers will favour the increase in revenue over the betterment of the public and responsible trading can easily be overlooked. 

For this reason the proposed solution is not to persuade the retailer to change their practice but the consumer. By equipping the consumer with an application that is able to monitor their inventory overstocking can be discouraged. The consumer will be driven by the incentive of monetary savings and as a side-effect reduce the risk of waste occurring at the consumer level. Theoretically creating an upstream ripple effect that will keep the waste at bay with the retailers and forcing the termination of over production. 

\subsubsection{System Overview}
The proposed system is comprised of two key components. The first is the RFID client. An RFID reader will reside in the fridge and monitor the presence of any tags in its vicinity, facilitating the automatic registration when loading and unloading products from the fridge, keeping an accurate inventory of the users contents. The user need not alter any previous behaviour. The second component is the mobile client that the consumer is able to access any time any place.

\subsection{Mobile \& Ubiquity}
As popularity in Smartphones increase many applications have identified the need for ubiquitously accessible counterpart software. Whether it is in the form of a web form, such as the popular taxi booking application Uber or by using available mobile platforms such as IOS for Apple or Android, the way in which we utilise software has shifted from the traditional static machines to mobile systems. The convenient size and connectivity has enabled Smart devices to evolve at a rapid pace and daily tasks such as checking our emails can now be completed on the go. But with the increase in popularity an over crowded market arises. A poorly functioning design can result in instant negative feedback displayed for all the world to see impacting whether others choose to download the application or not. Thus a sophisticated and responsive application is a high priority when developing such applications. 

\subsection{Automated Identification}
As discussed in the previous chapter, the primary draw back of existing systems such as the smart fridge and the mobile applications is the lack of automation in the item registration process. Convincing consumers to alter habitual behaviours such as stocking the fridge is challenging and would most likely be met with resistance unless the newer process yields better results than the current. Thus the user having to either manually punch in the product name, or scan every item with a barcode reader is an unappealing process and can prevent the user from utilising the technology. 


\subsubsection{Current Item Level Product Identification Methods} 
Predominantly, the current methods of item level identification in the supply chain rely on the ?picket-fence? style, one-dimensional barcode. add image Barcodes have been prevalent in grocery stores since the 1970?s and introduced as a means to manage inventory and to accelerate the checkout process. A laser scans the code and the catalogued code is used to retrieve detailed data such as the item name and cost of the product from the database. The scanning of the barcode requires that no objects obstruct the line of sight between the code and laser and even the smallest scratch or dirt can render the code unreadable. Simultaneous reads are not possible and must be scanned one by one with precision. Barcodes have limited capacity; usually can only hold twenty to twenty-five characters. Furthermore, the code is limited to identifying a collection of items rather than the individual item level, omitting crucial individual information such as sell-by-dates and fine-grain information of the produce. 

\subsubsection{Proposed Item Level Product Identification}
The proposed method of item-level product identification is by utilising RFID technology. RFID utilises electromagnetic waves emitted by a reader. The tag is comprised of a coil and transistor that harvests the energy from the reader. Once the tag has enough power the data is transmitted and the information is read from the tag. 

NFC registration and stock keeping systems have been successfully installed in retail stores, in particular clothing retailers. Before, logging of stock was a manual task. Employees would walk around the store floor with a laser reader and scan each item one by one. If the barcode had been damaged they would have to note the reference number down to enter into the system at a later time. With the use of RFID the employee now holds a reader and walks by the rack of garments, where the tags are read simultaneously, and the details captured with ease. Some grocery stores also use NFC but the technology is reserved for valuable goods to serve as an anti-theft device. Tags are applied in the form of a sticker and when detected by the readers, normally situated at the store exits, it will trigger an alarm to notify staff.

NFC has a few advantages over the traditional method of product identification. Firstly the tag does not require a clear line of sight. The reader may interrogate any tag within the vicinity of the transmitted electromagnetic field. The energy is passable through objects and some readers can even facilitate simultaneous reads. Secondly, tags are more durable than the traditional printed kind and can be embedded within the packaging or inconspicuously applied to the product. NFC also has more space to hold data.

With the use of RFID the consumer need not alter any behaviour. The item registration process can be completely automated removing the need for the consumer to manually enter product information or line up lasers with codes. Some other benefits can also be experienced through the application of item-level RFID, although out of the scope of the project the following case studies can highlight the benefits of item-level RFID identification in the food supply chain.

\subsubsection{Case Studies}
\paragraph{Transparency \& Consumer Rights} Even though producs are same in brand fresh goods can differ from batch to batch. Form the diets the cattel were on to the conditions the crops were exposed to such as pestisided. All this has an effect on our helth.
Food labels provide limited information of the produce we are consuming and have even been blamed for the lack of information given to consumers and that's why only minimal information is given.  

\subsection{Requirements Definition Report}
A requirement definition report is provided to support the development of the new system. Here the requirements are organised into categories following the UML recommended presentation as described in the book, ?System Analysis and Design with UML? by Tegarden and Wixom. (Suggestion was taken from the Information Systems module as part of the MSc Computer Science.)

\subsubsection{Non-Functional Requirements}
\underline{Operational Requirements}
\begin{itemize}
  \item The mobile element system should be able to operate in an Android environment. 
  \item The system should be portable and accessible from a mobile phone.
  \item The system should persist and read data from the database. 
  \item The system should scan and register RFID tags.
\end{itemize}
\underline{Performance Requirements}
\begin{itemize}
  \item The user interface must constantly be active.
  \item The system should be available for use 24 hours per day, 365 days per year.
  \item Any communication between the clients and the server must not exceed 5 seconds.
  \item The system should be durable and preserve data integrity.
\end{itemize}
\underline{Security Requirements}
\begin{itemize}
  \item Only authorised users can use the system.
  \item Users may only see the content of the subscribed fridge.
\end{itemize}

\vspace{\baselineskip}

\subsubsection{Functional Requirements}
\underline{Multi-user Support}
\begin{itemize}
  \item Any user should be able to join an existing fridge by entering the unique code provided to the primary owner of the fridge.
\end{itemize}
\underline{Information Viewing}
\begin{itemize}
  \item User should be able to view a list of their stocked items.
  \item User should be able to view the expiry date of the items with ease. 
  \item User should be able to see the quantity for each item.
  \item User should be notified when an item is added to their inventory.
  \item Users may only see the content of the subscribed fridge.
\end{itemize}
\underline{User Registration}
\begin{itemize}
  \item Users should be able to login to the system. 
  \item User should be able to logout by pressing the logout button.
\end{itemize}



\subsection{Technology \& Risk Analysis}
It has been estimated that 68\% of information technology projects fail, with this unfavourable statistic careful analysis and risk assessment of each technology is necessary to minimise the likelihood of encountering problems. Below analysis of each major component has been carried out.

\subsubsection{Mobile Platforms}
Currently the most used mobile platforms are Android and Apple IOS. Globally, the most widely used platform is Android, boasting an estimated user base of one billion users. 

IOS applications are written in languages such as Objective C, which requires the developer to manage the memory of objects. Recently a new language, Swift was introduced by Apple. Swift supports automatic reference counting (ARC), this eliminates large memory leaks that frequently occurred in it?s predecessor Objective C. Android on the other hand, has a high-performance memory management system that cleans up unused objects, also know as the garbage collector, allowing the developer to focus their time and attention on the functionality of the application. Android applications are written in Oracle Java. It is also note worthy to mention that Android has a unique ecosystem unlike that of a typical Java application. The developer must understand and abide by a certain rules of the environment and failing to do so can result in undesired outcomes. 

Swift being a relatively young language meant that less support is available compared to the others increasing the risks of encountering difficulties. Objective C will require diligent memory management and can shift focus from the core of the application and problems may arise from the lack of personal expertise.

Having previous exposure to Java, as it was taught as a core module (Programming in Java), the Android development environment was a sensible option. The familiarity of the language and integrated memory management minimised the risk of encountering difficulties. Android also has a larger user base making the application accessible to a wider audience.

\subsubsection{RFID}

Currently used in xyz, provide examples. Different classifications. 

\subsubsection{Cloud Services} 
Cloud services have gained notoriety due to the I.T giants Google, Amazon and Microsoft. The underlining principals of cloud computing of automated scalability and constant availability have converted the way I.T businesses operate. Enabling small organizations to begin developing with minimal capital expenditure. Traditional in house server farms and systems required thousands of pounds to for initial setup and maintain but with the delegation of responsibility to a cloud vendor, software and hardware is in effect rented, allowing businesses to focus on developing the application. To further emphasis, more businesses are adopting the agile methodology where change is to be expected and must be embraced, cloud services facilitates change by only charging the user for the resources consumed. Giving teams agility to react to change efficiently with minimal financial risk.

It?s highly responsive nature to demand and pay as you go style billing suites this project perfectly. Due to the unpredictable peaks and troughs in usage much like online retailers were the system for the most part experience steady traffic, but with the possibly of sudden spikes in demand during the holiday seasons and sales when consumers are most active. If the system cannot cope with the traffic and crashes it can cause severe damage to the business as an unresponsive application can cause users to go elsewhere. 

Out of the three leading cloud service Google Cloud Platform (GCP), Amazon Web Services (AWS) and Microsoft Azure the most fitting is AWS. Azure is recommended for those who are currently utilising the Microsoft stack such as C\# and MySql neither of which is relevant for this project. GAE?s data analytics engine has been praised as superior to that of AWS owed to the massive amounts of data harvested by Google. But in terms of services available AWS is without a doubt the leading vendor (see image).  AWS also has higher availability with eleven geographical locations that resources can be distributed to and users can efficiently access in comparison Google has only three regions. Although a notable disadvantage with AWS?s myriad of services is grasping the understanding of the complex network topology creating a steeper learning curve to GCP. Although GCP is simpler and user friendly it has been concluded that the positive elements of AWS has outweighed the negative.

\subsubsection{Data Storage}
%\paragraph{DynamoDB is a NoSQL database created by Amazon. 
%The reason to NoSQL, why not relations 
%TWhy amazon created it then opened it to the public}

\clearpage

%..................................................
% Design
%..................................................
\section{Design}

\subsection{Design Strategy}

Activity diagrams
Sequence diagram

\subsection{Architecture Design} Describes hardware software and network infrastructure. 
\subsection{Interface Design} How the user will navigate through the application. Navigation methods such as buttons and settings.

\subsection{Database and File specification} Defines exactly what date is being stored and where they will be stored.

\subsubsection{NoSQL Database} 
DynamoDB is a NoSQL key, value store distributed database designed by Amazon. The original motivation behind this was to cope with the massive amounts of data Amazon produces. The primary principal for DynamoDB is high availability, with data being replicated across multiple nodes if a node eliminating the single point of failure. Heartbeat signals are sent to the. 
\subsubsection{Internal Device Storage} 

\subsection{Program Design}This defines the programs that need to be written and exactly what each program will do. 

\subsubsection{Delay Tolerant Networking}
\subsubsection{Distributed Authentication \& Access Control}
\subsubsection{Multi-platform Consistency}

%WHAT - Definition, history
%TWHY - Motivation what is the reason why i chose it.
%INTENT
%COMPARISON
%OUTCOME

%what cloud services to use. Google app engine and sales force. Azure.
%What is a cloud service: 

%Why did i choose it?
%Available anytime, how it thrives big data. Admals Law Gustofsons Law
%Failure tolerance. how cloud services replicate data across nodes. 
%Why is it so popular. 
%Cost. 
%AWS stack for mobile development.
%Security

\(This may for the conclusion or implementation.\)Security:

\subsubsection{Mobile Application} 
%Android. IOS. Acknowledge the existence of IOS but focus on android. 

\subsubsection{RFID Client} %Different classifications of NFC.

%...................................................
% Implementation
%..................................................
\section{Implementation}

\subsection{System Construction}
\subsection{Android} 

\subsubsection{Life Cycle of a Process}
java for android.

\subsubsection{User Experience Design and Responsiveness}

\subsubsection{Networking}

\subsubsection{Threads and concurrency management}
Design static or dynamic comparidant
Activity life cycle.
UI and threads

\subsection{Setting up Amazon Web Service}
%Amazon Web Services operates on a 'pay as you go' basis. Each account is billed only for the rescouces used. Typically most organisations employ multiple employees with various responsibilities and roles, such employers must have different access rights and privileges to resources to avoid unintentional modification of data. For instance a database administrator should have read-write and table creating deletion and creating access whilst a programmer may not need the full access right to the table. AWS handles this the use of Identity Authentication Management (IAM).

%Horror stories of people uploading their credentials on the web such as version control.
%AWS created Cognito to avoid users having to hardcode credentials into application that are intended to mass distribution

\subsubsection{Amazon Web Service IAM}


\subsubsection{Amazon Web Service Cognito}
%Cognito is part of the AWS mobile development kit. AWS have enforced the use of Cognito to avoid the need to hard code credentials into applications that are intended for mass distribution such as mobile applications. Cognito uses Identity Pools to authenticate a user. Identity pools have a security policy where the details of the policy is created. 

Example policy goes here

\subsection{User Authentication}
\subsubsection{Third-party Authentication}


%-facebook google iam roles
%cognito and the use of identity pool. 
%-Analytics, how these api's provide analytics, user patterns

\subsection{Data Management}
\subsubsection{Amazon Web Service DynamoDB}
%What is it? Key value, uses JSON serialisation. supports strings integers etc. 
%Limitations 4k blocks

%Discoveries?
%Outcome
\subsubsection{NoSQL model}
%TNoSQL comparison to relational database. 
\subsubsection{Data modelling} 
\subsubsection{Caching Data} offline use 

\subsection{RFID Client}%classification, rather than using the android phone i used an actual reader.

%...................................................
% Testing and evaluation
%..................................................
\section{Testing and Evaluation}
Importance, maintainable code.
\subsection{Unit test and mocking}
\subsection{Integration Test}

%...................................................
% Conclusion 
%..................................................
\section{Review and Conclusion}
\subsection{Security Concerns} look at link from George. 



\vspace{\baselineskip}
\vspace{\baselineskip}
\vspace{\baselineskip}


\end{document}
